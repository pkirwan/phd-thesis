% ************************** Thesis Abstract *****************************
% Use `abstract' as an option in the document class to print only the titlepage and the abstract.
\begin{abstract}

    Infectious diseases have plagued humanity for millennia, with outbreaks such as cholera and smallpox claiming countless lives. As was demonstrated by COVID-19, our interconnected world is now more susceptible than ever to the global spread of an infectious disease pandemic.

    Despite advances in statistical methods and comprehensive public health surveillance, understanding the many facets of infectious disease burden remains a challenge. Questions persist of how fast a disease is spreading, who in society is most at risk, and which interventions might have the capability to save lives.

    In this thesis I aim to develop and apply multi-state models to estimate infectious disease burden from observational data. Multi-state models are a flexible approach to estimating risks and lengths of stay, and capable of accounting for data biases and censoring. I demonstrate the potential of these models to address certain limitations of classical survival analysis methods, and their application to answer causal questions using observational data.

    Firstly, I develop an established multi-state back-calculation method to better estimate the incidence processes underlying the HIV epidemic in England. The updated method is shown, via simulation, to account for measurement bias in markers of disease progression. This method is also shown to be robust to interruptions in HIV testing activity and transmission which may have occurred as a result of COVID-19 lockdowns.

    Next, I apply multi-state models to patient admissions for COVID-19 to estimate hospitalised fatality risks and lengths of stay in hospital using a variety of linked hospital datasets. Changes in patient prognosis are estimated throughout each wave of infection, as well as the beneficial effect of vaccination, and the deleterious effect of hospital pressure on patient outcomes.

    Finally, I use multi-state models to estimate the real-world effectiveness of SARS-CoV-2 vaccination among a cohort of healthcare workers in England. These models are applied in a causal inference framework and a 14\% effectiveness of fourth dose vaccination against infection is estimated for this cohort, relative to a waned third dose.

    Multi-state models are a powerful tool for the estimation of infectious disease burden and have proven invaluable for research concerning the COVID-19 pandemic. The methods and applications in this thesis will inform future areas of research:\ accounting for population migration in HIV incidence estimates, developing an improved metric of hospital pressure, and assessing the effectiveness of future SARS-CoV-2 vaccine roll-outs.

\end{abstract}
