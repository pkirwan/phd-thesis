% ******************************* Impact statement ********************************

\begin{impact}\addcontentsline{toc}{chapter}{Impact statement}

    The overarching objective of this thesis has been to develop and apply statistical modelling approaches for public health data monitoring, thereby enhancing our collective understanding of ongoing disease epidemics and providing evidence for decision makers and the wider community. The incidence, burden of disease, and effectiveness of interventions are perhaps the three most important quantities of interest for managing any infectious disease. Improving the precision and accuracy of these metrics whilst accounting for biases in biomedical and public health data is, therefore, a goal for public health data science research. The specific contributions that this thesis has made to infectious disease research are two-fold:\ increasing knowledge for public health action, and methodological developments in monitoring disease epidemics.

    \subsubsection*{Increasing knowledge for public health action}

    I generated annual estimates of HIV incidence and undiagnosed HIV prevalence for national surveillance reports and the national action plan on HIV elimination, which have been used to inform HIV prevention campaigns and national policy decisions.

    My estimates of hospitalised fatality rates and lengths of stay in hospital for COVID-19 informed expert scientific panels mobilised during the initial months of the pandemic. Meanwhile, my estimates of SARS-CoV-2 vaccine effectiveness among a healthcare worker cohort informed decisions about national vaccine policy, specifically the roll-out of fifth booster vaccines to healthcare workers.

    \subsubsection*{Methodological developments in monitoring disease epidemics}

    I advanced established statistical methods in HIV estimation; developing and evaluating a revised definition of late HIV diagnosis, enhancing the precision of an existing multi-state model through the inclusion of additional biomarker data, and exploring the sensitivity of model estimates to counterfactual assumptions. The revised late HIV diagnosis definition has since become the standard approach for assessment of this measure, both in the UK and in Europe, and estimates from the dual biomarker method for HIV estimation continue to inform progression towards England's Action Plan on HIV\@.

    The COVID-19 pandemic necessarily required the application of novel methodologies to better understand the newly emerging virus and its sub-variants. I developed multi-state models of hospitalisation and vaccination which were tailored to the specifics of different surveillance systems, and extended vaccine effectiveness models in formal causal inference analyses.

    \subsubsection*{Research publications}

    The research presented in this thesis has been disseminated via three first-author publications in peer-reviewed journals, alongside several co-authored publications and conference presentations (see Appendix~\ref{sec:publications}). My work to increase our understanding of the emerging COVID-19 pandemic was recognised in 2022 with the Scientific Pandemic Influenza Group on Modelling (SPI-M)\nomenclature[z]{SPI-M}{Scientific Pandemic Influenza Group on Modelling} Award for Modelling and Data Support (SAMDS).

    \sectionbreak{}

    Aside from the contributions to research generated by this thesis, working on infectious disease modelling during a pandemic has provided me with the unique opportunity to undertake research whilst understanding of this novel pathogen was still developing. I was fortunate to collaborate with a wide range of statisticians, scientists, doctors, epidemiologists, public health professionals, and government advisors in the national and international response to COVID-19. The various projects I have been involved with have broadened my knowledge of infectious disease statistics and epidemiology, improved my understanding of the role of statistics in public health decision making, and enhanced my ability to communicate statistical concepts.

    Nevertheless, there has been an unavoidable impact of the pandemic on my PhD and this thesis is a combination of my original project focused on HIV back-calculation, and my subsequent research into COVID-19 severity and protection against infection. Whereas the HIV research had a clearer set of aims and well-defined methodology, the COVID-19 research was faster-paced, with an evolving research question, and recurrent data updates. Central to this research has been the application of statistical methods to estimate infectious disease burden, and this theme is explored in different ways throughout this thesis.

\end{impact}